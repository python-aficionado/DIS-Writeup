\pagenumbering{roman} 
\setcounter{page}{1} 
\pagestyle{plain}

%%%%%%%%%%%%%%%%
%%% CREDITOS %%%
%%%%%%%%%%%%%%%%
%\chapter*{Créditos/Copyright}

%Una página con la especificación de créditos/copyright para el proyecto (ya sea aplicación por un lado y documentación por el otro, o unificadamente), así como la del uso de marcas, productos o servicios de terceros (incluidos códigos fuente). Si una persona diferente al autor colaboró en el proyecto, tiene que quedar explicitada su identidad y qué hizo.

%A continuación se ejemplifica el caso más habitual, aunque se puede modificar por cualquier otra alternativa:

%\vspace{1cm}

% \begin{figure}[ht]
%     \centering
% 	\includegraphics[scale=1]{images/license.png}
% \end{figure}

% Esta obra está sujeta a una licencia de Reconocimiento -  NoComercial - SinObraDerivada

% \href{https://creativecommons.org/licenses/by-nc-nd/3.0/es/}{3.0 España de CreativeCommons}.

%%%%%%%%%%%%%
%%% FICHA %%%
%%%%%%%%%%%%%
% \chapter*{FICHA DEL TRABAJO FINAL}

% \begin{table}[ht]
% 	\centering{}
% 	\renewcommand{\arraystretch}{2}
% 	\begin{tabular}{r | l}
% 		\hline
% 		Título del trabajo: & Descriptivo del trabajo\\
% 		\hline
%         Nombre del autor: & Nombre y dos apellidos\\
% 		\hline
%         Nombre del colaborador/a docente: & Nombre y dos apellidos\\
% 		\hline
%         Nombre del PRA: & Nombre y dos apellidos\\
% 		\hline
%         Fecha de entrega (mm/aaaa): & MM/AAAA\\
% 		\hline
%         Titulación o programa: & Plan de estudios\\
% 		\hline
%         Área del Trabajo Final: & El nombre de la asignatura de TF\\
% 		\hline
%         Idioma del trabajo: & Catalán, español o inglés\\
% 		\hline
%         Palabras clave & Máximo 3 palabras clave\\
% 		\hline
% 	\end{tabular}
% \end{table}

%%%%%%%%%%%%%%%%%%%
%%% DEDICATORIA %%%
%%%%%%%%%%%%%%%%%%%
% \chapter*{Dedicatoria/Cita}

% Breves palabras de dedicatoria y/o una cita.

%%%%%%%%%%%%%%%%%%%
%%% Agradecimientos %%%
%%%%%%%%%%%%%%%%%%%
% \chapter*{Agradecimientos}

% Si se considera oportuno, mencionar a las personas, empresas o instituciones que hayan contribuido en la realización de este proyecto.

%%%%%%%%%%%%%%%%
%%% RESUMEN  %%%
%%%%%%%%%%%%%%%%
\chapter*{Abstract}
\addcontentsline{toc}{chapter}{Abstract}

\onehalfspacing

Question-answering systems for healthcare have been widely implemented, so in the specific area of breastfeeding, where women may not have access to relevant, convenient, on-demand information an automated QA system could prove a valuable resource. With that aim in mind, using expert generated content and user submitted queries, a QA system will be built using state-of-the-art techniques and technologies, leveraging pre-trained models, open source algorithms, and commercially available solutions where relevant. 

\vspace{1.5cm}

Els sistemes de resposta a preguntes per a l'assistència sanitària s'han implementat àmpliament, de manera que en l'àrea específica de la lactància materna, on les dones poden no tenir accés a informació rellevant, convenient i immediata, un sistema de pregunta-resposta automatitzat podria crear valor per les usuàries. Amb aquest objectiu en ment, utilitzant contingut generat per experts i consultes enviades per les usuàries, es construirà un sistema de pregunta-resposta utilitzant tècniques i tecnologies d'última generació, aprofitant models pre-entrenats, algorismes de codi font obert i solucions comercials quan sigui necessari.
\vspace{1.5cm}

\textbf{Keywords}: Information Retrieval, Question-Answering, Breastfeeding, Natural Language Processing